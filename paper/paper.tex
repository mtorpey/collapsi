\documentclass[a4paper, twocolumn]{article}

\usepackage{amsmath}
\usepackage{fullpage}
\usepackage[hidelinks]{hyperref}
\usepackage[UKenglish]{isodate}
\usepackage{tcolorbox}
\usepackage{url}

\tcbset{height=0.9cm, width=0.8cm, valign=center, halign=center, left=0cm, right=0cm}
\setlength{\tabcolsep}{0.1cm}
\newcommand\card[1]{\begin{tcolorbox}#1\end{tcolorbox}}

\title{Collapsi is strongly solved}
\author{Michael Young\\University of St Andrews}

\begin{document}

\maketitle

\abstract{ Collapsi is a two-player game of complete information released in
  June 2025 by Mark~S.~Ball of \emph{Riffle Shuffle \& Roll}. Played with two
  pawns on a torroidal board of 16 randomly mixed playing cards, players take it
  in turns to move based on the value of the card they sit on, with the game
  ending when a player has no legal moves.

  The number of possible deals after symmetry breaking is low enough, and the
  game tree is shallow enough, to make an exhaustive analysis of the game
  feasible. A search was applied revealing that the first player can force a win
  in 37.8\% of deals, with the second player able forcing a win in all others. In
  6.3\% of deals the losing player can prolong the game to the maximum length of
  14 plies, and a win can never be forced in less than 7 plies.
}


\section{The game}

The rules for Collapsi are hosted by the designer in a living document
\cite{rules} and in the game's introductory YouTube video \cite{youtube}. The
board is a set of 16 playing cards taken from a standard deck -- four aces, four
2s, four 3s, two 4s and two jokers -- shuffled and arranged into a $4\times 4$
grid, an example layout being shown in Figure \ref{fig:board}. Each player takes
a pawn (red or blue) and places it on a joker, then players take turns
(\textit{plies}) to move with red going first.

\begin{figure}[h]
  \centering
  \begin{tabular}{c c c c}
    \card{A} & \card{2} & \card{2} & \card{3} \\
    \card{4} & \card{A} & \card{2} & \card{J} \\
    \card{3} & \card{A} & \card{2} & \card{3} \\
    \card{J} & \card{3} & \card{A} & \card{4}
  \end{tabular}

  \caption{Example Collapsi board layout}
  \label{fig:board}
\end{figure}

On a player's first turn they may choose to move 1, 2, 3 or 4 spaces; on
subsequent turns they must move exactly the number of spaces shown on the card
they begin from, with ace representing 1.  (TODO: change of rule for jokers?)
Moving is done orthogonally and the grid is torroidal so that the top edge is
joined to the bottom, and the left to the right. On a given move, a pawn may not
enter a given space twice, and may not end its movement on the space occupied by
the opponent's pawn.

After a player's ply, the card they began from is turned face down and cannot
be entered for the rest of the game. The first player who cannot make a legal
move loses the game, which must therefore happen in at most 14 plies.


\section{Game length}




\section{Search space}

To find a solution, all possible arrangements of the 16 cards must be
considered, naively a space of $16! = 2.1 \times 10^{13}$ possible
arrangements. However, various symmetries allow us to reduce the search space by
eliminating deals that are strategically equivalent, resulting in a tractable
number of deals that need to be considered.

First observe that suit plays no part in the game: all 3s are equivalent, and so
on. Note also that a player's start position has no effect, since a move of 1--4
allows them to reach any uncovered space on their first ply; this means the two
jokers can be considered equivalent.

The torroidal nature of the board means that the bottom row can be moved to the
top, or the right column moved to the left edge, without materially affecting
play. Each of these operations can be cycled, so that any row can be chosen as
the top one and any column as the left one. We model this by only considering
boards with a joker in the top-left corner, without loss of generality.

We can also constrain the position of the second joker without loss of
generality, to take advantage of reflections in the board. Two possible
positions for the second joker are strategically equivalent if the $(x, y)$
distance between the first joker and the second is the same in both cases,
possibly including an $x$--$y$ swap. The possible distances are shown in
Figure~\ref{fig:joker-distances}, with one representative of each distance
highlighted: these 5 positions are the positions considered for the second joker
when searching.

\begin{figure}[h]
  \centering
  \begin{tabular}{c c c c}
    \card{J} & \card{\textbf{0,1}} & \card{\textbf{0,2}} & \card{0,1} \\
    \card{0,1} & \card{\textbf{1,1}} & \card{\textbf{1,2}} & \card{1,1} \\
    \card{0,2} & \card{1,2} & \card{\textbf{2,2}} & \card{1,2} \\
    \card{0,1} & \card{1,1} & \card{1,2} & \card{1,1}
  \end{tabular}

  \caption{Distances of spaces from top-left joker}
  \label{fig:joker-distances}
\end{figure}

Overall, with the position of one joker fixed, 5 options for the second joker,
and the remaining 14 cards split into three sets of four with the two 4s left
over, the number of deals that must be considered is equal to
$$5~\binom{14}{4} \binom{10}{4} \binom{6}{4} = 15~765~750,$$
a number much more amenable to search than the naive space initially considered.

Different deals have game trees of different sizes, but informal experiments
show that around 900,000 possible games can be played from a typical deal.


\section{Computation}

A Rust library for enumerating and exploring Collapsi games was written
\cite{github} and exhaustive experiments performed using a 13th Gen Intel Core
i5-13500 processor. All 15~765~750 deals can be enumerated in 3.3 seconds and
then explored in parallel.

The algorithm used for evaluating game positions was minimax search with
alpha--beta pruning and unlimited depth. This was applied both for the simple
goal of finding a winning move, and for the stricter goal of maximising score
with respect to game length.

Given a new board, it can be determined whether the current player is in a
winning position, and if so a winning move can be determined, in around 12
milliseconds. To find a move that maximises the current player's score with
respect to turn length takes slightly longer, around 18 milliseconds. These very
short computation times suggest that there would be little value in the creation
of a database of game positions, since the results can be computed on demand for
most relevant applications.

The result with game-length perfect play was determined for all deals in a
search that took 7 hours and 29 minutes using 20 cores. All results can be
reproduced by following the instructions provided in the library's readme file
\cite{github}.


\section{Results}

The length of games given game-length-perfect play from both sides is shown in
Table~\ref{tab:game-length}.

\begin{table}[h]
  \centering
  \begin{tabular}{r r r c}
    \hline
    Plies & Deals & (\%) & Winner \\
    \hline
    $\leq 6$ & 0 & 0.0 & ---\\
    7 & 8 & 0.0 & Red \\
    8 & 62604 & 0.4 & Blue \\
    9 & 30470 & 0.2 & Red \\
    10 & 1156548 & 7.3 & Blue \\
    11 & 1869785 & 11.9 & Red \\
    12 & 7590554 & 48.1 & Blue \\
    13 & 4054671 & 25.7 & Red \\
    14 & 1001110 & 6.3 & Blue \\
    \hline
  \end{tabular}
  \caption{Length of game with perfect play}
  \label{tab:game-length}
\end{table}



\section{Future work}

In late June 2025, Collapsi's designer offered an updated set of rules for the
game, described as version 1.2.0, with minor modifications to the rules for
jokers. The library should be expanded to allow for analysis of the new game,
which may differ in its theoretical status.

The game exploration was performed exhaustively without consideration for how
this solution could inform human play. Now that a library exists and perfect
moves can be retrieved easily, its moves should be considered qualitatively,
since they might highlight strategies that players could follow without deep
analysis to improve their play, especially in the early game where it can be
difficult to understand the value of a move.


\begin{thebibliography}{9}

\bibitem{rules}
  Mark~S.~Ball,
  \textit{Collapsi 1.1.0},
  \url{https://boardgamegeek.com/filepage/301918/collapsi-official-rules}
  %\url{https://docs.google.com/document/d/1MV1vvV1Aq7ikRU7lF8Y7QQD-akH3ybHmk2fKs6p61lE}
  (Accessed 2025-06-30).

\bibitem{youtube}
  Riffle Shuffle \& Roll,
  \textit{How to Play Collapsi: NEW two player abstract game with playing cards!}
  YouTube video,
  \url{https://youtu.be/6vYEHdjlw3g}
  (Accessed 2025-06-30).

\bibitem{github}
  M.~Young,
  \textit{collapsi-solver},
  Rust crate,
  2025,
  \url{https://github.com/mtorpey/collapsi-solver}

\end{thebibliography}

\end{document}